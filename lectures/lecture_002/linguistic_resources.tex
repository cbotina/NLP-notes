\section{Linguistic Resources}

The implementation of different Natural Language Processing tasks requires various linguistic resources. Among the linguistic knowledge bases, the following stand out:
\begin{itemize}
    \item Dictionaries
    \item Lexicons
    \item Thesauri
    \item Lexical relation databases
\end{itemize}

Additionally, collections of texts or speech resources called \textbf{corpora} (singular: \textbf{corpus}) are also important linguistic knowledge bases.

\subsection{Dictionary}

A \textbf{dictionary} is a repertoire where words of a language are grouped according to a specific order, accompanied by their definition or explanation. Dictionaries include a detailed and human-understandable description of the different senses of words.

\textbf{Example:} A dictionary entry for ``banco'' might look like:
\begin{quote}
\textbf{banco} \textit{(noun)} \\
1. A financial institution that accepts deposits and makes loans. \\
2. A long seat for several people, typically made of wood or stone. \\
3. A group of fish swimming together.
\end{quote}

\subsection{Lexicon}

A \textbf{lexicon} is a dictionary that contains morphological information. This morphological dictionary is a repertoire that collects a list of morphemes and basic information about them.

\textbf{Example:} A lexicon entry might include:
\begin{quote}
\textbf{cantábamos} \\
Lexeme: \textit{cant} \\
Morphemes: \textit{cant} + \textit{aba} + \textit{mos} \\
Morphological information:
\begin{itemize}
    \item Root: \textit{cant-} (to sing)
    \item Tense/Aspect: \textit{-aba-} (imperfect past)
    \item Person/Number: \textit{-mos} (first person plural)
\end{itemize}
\end{quote}

\subsection{Thesaurus}

Also called \textbf{thesaurus}, \textbf{thesauri}, or \textbf{tesoro}. It refers to a dictionary that contains a list of word meanings and the relationships between these meanings.

\textbf{In literature:} A thesaurus contains a list of words with their synonyms and antonyms.

\textbf{In linguistics:} A thesaurus gathers knowledge about hypernymy/hyponymy and meronymy/holonymy relationships represented in the thesaurus's hierarchical structure. That is, the thesaurus hierarchy models:
\begin{itemize}
    \item The \textbf{is-a} relationship (hypernymy/hyponymy)
    \item The \textbf{part-whole} relationship (meronymy/holonymy)
\end{itemize}

Additionally, a thesaurus can also group knowledge about other semantic relationships such as synonymy or antonymy.

\textbf{Example (Literature):} A thesaurus entry for ``happy'':
\begin{quote}
\textbf{happy} \\
Synonyms: joyful, cheerful, content, delighted, pleased \\
Antonyms: sad, unhappy, miserable, depressed
\end{quote}

\textbf{Example (Linguistics):} A hierarchical thesaurus structure:
\begin{quote}
\textbf{vehicle} (hypernym) \\
\quad $\rightarrow$ \textbf{car} (hyponym) \\
\quad \quad $\rightarrow$ \textbf{sedan} (hyponym) \\
\quad \quad $\rightarrow$ \textbf{SUV} (hyponym) \\
\quad $\rightarrow$ \textbf{bicycle} (hyponym) \\
\\
\textbf{car} (holonym) \\
\quad $\rightarrow$ \textbf{wheel} (meronym) \\
\quad $\rightarrow$ \textbf{engine} (meronym) \\
\quad $\rightarrow$ \textbf{door} (meronym)
\end{quote}

\begin{tcolorbox}[colback=blue!5!white, colframe=blue!75!black, title=\textbf{Curious Fact: Semantic Relationships}]
\textbf{Hypernym and Hyponym} (is-a relationship):
\begin{itemize}
    \item A \textbf{hypernym} is a more general term that encompasses more specific terms. Example: ``vehicle'' is a hypernym of ``car''.
    \item A \textbf{hyponym} is a more specific term that belongs to a broader category. Example: ``car'' and ``bicycle'' are hyponyms of ``vehicle''.
    \item Relationship: ``A car \textit{is a} vehicle'' (hyponym is-a hypernym).
\end{itemize}

\textbf{Holonym and Meronym} (part-whole relationship):
\begin{itemize}
    \item A \textbf{holonym} is a whole that contains parts. Example: ``car'' is a holonym of ``wheel''.
    \item A \textbf{meronym} is a part that belongs to a whole. Example: ``wheel'', ``engine'', and ``door'' are meronyms of ``car''.
    \item Relationship: ``A wheel \textit{is part of} a car'' (meronym part-of holonym).
\end{itemize}

\textbf{Mnemonic:} Think of hypernym as ``super'' (above) and hyponym as ``sub'' (below) in a hierarchy. For holonym/meronym, remember that a ``whole'' (holonym) contains ``parts'' (meronyms).
\end{tcolorbox}

\subsection{Lexical Relation Databases}

A \textbf{lexical relation database} is a generalization of thesauri---a type of dictionary that collects knowledge about any semantic relationship between word senses. These databases contain:
\begin{itemize}
    \item A set of lemmas
    \item Each lemma annotated with:
    \begin{itemize}
        \item The possible set of senses of the word
        \item The relationships between those senses
    \end{itemize}
\end{itemize}

\textbf{Example:} A lexical relation database entry (similar to WordNet structure):
\begin{quote}
\textbf{Lemma:} \textit{car} \\
\textbf{Synset 1:} \{car, auto, automobile, machine, motorcar\} \\
\quad Sense: ``a motor vehicle with four wheels'' \\
\quad Hypernym: vehicle \\
\quad Hyponyms: sedan, SUV, coupe, convertible \\
\quad Meronyms: wheel, engine, door, window \\
\quad Part-of: traffic, fleet \\
\\
\textbf{Synset 2:} \{car, railcar, railway car, railroad car\} \\
\quad Sense: ``a wheeled vehicle adapted to the rails of railroad'' \\
\quad Hypernym: wheeled vehicle \\
\quad Hyponyms: passenger car, freight car, dining car
\end{quote}

\subsection{Linguistic Corpus}

A \textbf{linguistic corpus} is a collection of texts representative of a language used for linguistic analysis.

\subsubsection{Types of Corpora}

\textbf{Textual corpora:} Collections of extracts from books, magazines, newspapers, or any other written source.

\textbf{Oral corpora:} Collections of speech, i.e., audio extracts from sources such as radio or television.

\subsubsection{Tagged Corpora}

Corpora can be \textbf{annotated} or \textbf{tagged} so that the words they contain present additional linguistic information, such as:
\begin{itemize}
    \item Syntactic information
    \item Semantic information
    \item Pragmatic information
\end{itemize}

These are known as \textbf{tagged corpora}.

\subsubsection{Historical Development}

The first corpora available online appeared in the 1960s. One of the pioneers was the \textbf{Brown Corpus}, an American English corpus developed in 1963 by Brown University, containing a collection of one million words extracted from 500 texts of different genres: newspapers, novels, non-fiction, academic, etc. (Kucera and Francis, 1967). Initially, the corpus was not tagged, but over the years it was tagged with information about grammatical categories (POS).

\textbf{Modern corpora:} Today's corpora tend to be much more extensive than the Brown Corpus. For example, the \textbf{WSJ corpus} (Wall Street Journal), tagged with morphosyntactic information, is a collection of one million words published in Wall Street Journal articles in 1989.


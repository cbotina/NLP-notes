\section{Morphology}

According to the Royal Spanish Academy (RAE), \textbf{morphology} is ``the part of grammar that studies the structure of words and their constituent elements''.

Morphology studies how words decompose into indivisible parts, each with meaning. These minimal units are called \textbf{morphemes}.

\begin{definition}[Morpheme]
The minimal unit that composes a word and is capable of expressing meaning.
\end{definition}

\textbf{Example:} The word ``mujeres'' (women) contains two morphemes:
\begin{itemize}
    \item ``mujer'' (woman) --- the root
    \item ``es'' --- indicates plural
\end{itemize}

\subsection{Grammatical Morphemes and Lexemes}

A \textbf{grammatical morpheme} has grammatical meaning, such as:
\begin{itemize}
    \item Gender (masculine or feminine)
    \item Number (singular or plural)
    \item Person (e.g., third person singular)
    \item Mood and tense (e.g., indicative mood, future tense)
\end{itemize}

In contrast, a \textbf{lexeme} is the morpheme that forms the root or invariant part of the word, providing the main lexical meaning.

A \textbf{lemma} is the canonical form of a word (e.g., infinitive for verbs, singular for nouns) used as the dictionary entry.

\begin{definition}[Lemma]
The canonical or dictionary form of a word, used as the base form for all its inflected variants.
\end{definition}

\textbf{Example:} In ``mujeres'':
\begin{itemize}
    \item ``mujer'' is the lexeme (root with lexical meaning)
    \item ``es'' is the grammatical morpheme (expresses plural number)
    \item ``mujer'' is also the lemma (canonical form)
\end{itemize}

The singular form ``mujer'' contains:
\begin{itemize}
    \item The lexeme ``mujer''
    \item A \textbf{zero morpheme} for singular number
\end{itemize}

\begin{definition}[Zero Morpheme]
A grammatical morpheme without phonetic realization but with grammatical meaning.
\end{definition}

\subsection{Example: Complex Word Structure}

The word ``cantábamos'' (we were singing) contains three morphemes:
\begin{enumerate}
    \item \textbf{Lexeme ``cant''}: Lexical meaning (the act of producing melodious sounds with the voice)
    \item \textbf{Grammatical morpheme ``aba''}: Indicative mood, past tense
    \item \textbf{Grammatical morpheme ``mos''}: First person plural
\end{enumerate}

The lemma for ``cantábamos'' is ``cantar'' (to sing).

\subsection{Morphological Analysis Examples}

Table~\ref{tab:morphology_examples} shows examples of morphological analysis, decomposing words into their lexemes, grammatical morphemes, and lemmas.

\begin{table}[H]
\centering
\begin{tabular}{lcccc}
\toprule
\textbf{Word} & \textbf{Lexeme} & \textbf{Gram. Morpheme (1)} & \textbf{Gram. Morpheme (2)} & \textbf{Lemma} \\
\midrule
Cantábais & Cant & ába & Is & Cantar \\
Casita & Cas & Ita & Zero (SG) & Casa \\
Perros & Perr & O & S & Perro \\
Vino & Vino & Zero (M) & Zero (SG) & Vino \\
\bottomrule
\end{tabular}
\caption{Examples of morphological analysis in Spanish}
\label{tab:morphology_examples}
\end{table}

\subsection{Computational Morphology}

\textbf{Computational morphology} aims to automatically recognize the morphemes contained in a word.

\textbf{Importance in NLP:}
\begin{itemize}
    \item \textbf{Search and retrieval}: When searching for ``pensar'' (to think), also recognize ``piénsalo'' (think about it), even though they don't share the same character sequence in the root
    \item \textbf{Text generation}: Ensure grammatical agreement (e.g., gender and number between nouns and adjectives, person between verbs and subjects)
\end{itemize}

\subsection{English vs. Spanish Morphological Analysis}

Morphological analysis in English (where most computational morphology research is conducted) is simpler than in Spanish because:
\begin{itemize}
    \item Spanish has many more variations in word endings
    \item Spanish has more root alterations (stem changes)
\end{itemize}

